The aim of this project is using the protocol analyzer \textit{Wireshark} for classifying traffic at \textit{MAC} layer.\\ 
To reach this goal, we used \textit{Wireshark} in \textit{monitor\ mode}: in this mode, the \textit{WiFi} module of our PC
intercepts all the traffic in range that is transiting via \textit{WiFi}.\\ 
This is a limitation for our project in the sense that at \textit{MAC} layer it's possible to check whether a frame is transporting data, but it's 
not possible to distinguish the application that has generated the data carried in the frame payload. Therefore, our 
analysis only provides a quantitative estimation of the traffic of data transported over \textit{WiFi}. The reason why we have 
used this solution even if we were aware of this limitation is that working in monitor mode is the only solution we had
to sniff traffic that was not meant only for the device that was performing the capture.\\

\subsection{{Our analysis}}
    In this project, what we have done is looking at the \textit{data} and \textit{QoS data} frames captured by our computers and
    try to understand what those packets are meant to in terms of type of application that has generated them. As specified, this 
    can only be a deduction: we can't be sure of the application that has generated a packet at \textit{MAC} layer.\\ 
    To do this we have scanned many sample captures and for each one of them:
    \begin{enumerate}
        \item Found all the \textit{MAC} addresses that have generated/received at least one \textit{data/QoS\ data} frame.
        \item Associated to all the \textit{MAC} addresses we have revealed:
        \begin{itemize}
            \item The number of transmitted/received bytes.
            \item The number of transmitted/received packets.
            \item The average uplink/downlink rate.
        \end{itemize}
        \item Found the vendor associated with the \textit{MAC} address.
    \end{enumerate}
    In order to analyze the captured traffic and try to understand the type of application a user (so a \textit{MAC}) was using,
    we have plotted: 
    \begin{itemize}
        \item A pair of histograms (one for the bytes, one for the number of packets) in which there is a bar for each
                \textit{MAC} address that have generated/received at least a certain amount of packets (better description of
                this later on).
        \item A graph for the cumulative uplink/downlink traffic (considering then all the bytes generated/received by all the 
                revealed \textit{MACs}).
        \item A graph for the discrete traffic in uplink/downlink, that is the total amount of bytes exchanged by all the revealed
                \textit{MACs} in a fixed time interval (customizable in the program).
    \end{itemize}