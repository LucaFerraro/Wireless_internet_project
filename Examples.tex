% Descrivere cosa c'è di interessante nella cattura evidenziando i MAC che conosciamo, quelli che 
% fanno cose che abbiamo capito (ad esempio i sonos oppure i dispositivi che usano zoom) e inserire
% le immagini come contorno

<<<<<<< HEAD
To test our program we did some captures in different moments of the day to see what we are able to 
understand from it.
\subsection{Sonos speakers interconnected via WiFi}
=======
\subsection{Sonos capture.}
>>>>>>> 370de4e35883f001c1c6d3fcf0493189a25769a1
%Fabio
 A capture that brought something interesting to our attention is the \texttt{Filtered
\_capture\_SONOS\_WEDNESDAY\_MILAN.pcapng} done in Milan on a wednesday morning. Thanks to the vendor 
classification we spotted a four MAC addresses related to \textbf{Sonos} devices, a vendor famous for
wireless home sound systems in which multiple speakers are interconnected using WiFi. In this capture 
we found four different devices which produced most of the traffic, the MAC addresses of the Sonos 
devices are:
\begin{itemize}
    \item 00:0e:58:c2:48:5f
    \item 00:0e:58:c2:48:05
    \item 00:0e:58:67:a3:e9
    \item 00:0e:58:f4:7a:83
\end{itemize}
As we can see in Figure \ref{fig:Sonos_traffic}, there are a lot of traffic spikes mainly due to the
communication between those speakers. The highest peak is the combination of the communication between
the Sonos devices and other communication flows.
\begin{figure}[h]
    \includegraphics[width=\textwidth]{Graphs/SONOS_cum_in_traffic.png}
    \caption{Sonos input traffic graphs}
    \label{fig:Sonos_traffic}
\end{figure}
\\
To further show how many packets the Sonos devices exchanged we can see that three out of the four
devices are at the top spots of the following graph.
\begin{figure}[h]
    \includegraphics[width=\textwidth]{Graphs/SONOS_bytes_packets.png}
    \caption{Sonos packets exchanged}
    \label{fig:Sonos_packets}
\end{figure}

<<<<<<< HEAD

 
\subsection{ZOOM}
=======
\subsection{Zoom captures.}
>>>>>>> 370de4e35883f001c1c6d3fcf0493189a25769a1
%Fabio

\subsection{Multimedia internet capture.}
In this paragraph, we analyze the results obtained with a capture performed during a lecture of
multimedia internet. \\ 
The lecture has been followed using \textit{Microsoft\ Teams}: the professor was presenting the
topic sharing its screen and all the students were following him with camera and microphone inactive.\\
In the Figure \ref{Multimedia internet lecture: data and packets exchanged.}, we can see on the
left the graphs reporting the number of bytes transmitted and received by some of the \textit{MACs}
we have revealed and on the right the number of transmitted and received by the same \textit{MACs}.\\
As we can see, there are some \textit{MACs} that are exchanging much more traffic than others:

\begin{itemize}
    \item \textbf{88:ae:07:3d:8a:30}: this is the device (an \textit{iPad\ Pro}) from which the lecture 
            was being attended. Even if the capture lasted only 5 minutes, we can see that this 
            device has received a lot of bytes (about 42.660 MB): this is of course due to the fact
            that the device was using a video-conferencing application.\\ 
            If we consider the number of packets received by this \textit{MAC}, we can compute the 
            average length of the packets during the 5 minutes of the capture:

            \begin{equation}
                \textit{Average packet length} = \textit{Number of bytes} / \textit{Number of packets} = 594 Bytes
            \end{equation}

    \item \textbf{20:b0:01:22:22:66}: this is the access point of the home were the lecture was being
            attended. Indeed, we can see that the access point has sent a lot of bytes and packets: 
            most of them were probably destined to the \textit{iPad\ Pro} that was being used to
            follow the lecture, since that device was connected to this access point while attending 
            the class. 
\end{itemize}

Among the other \textit{MAC} addresses present in the graph but that have exchanged only a few 
packets in compared with the 2 ones mentioned above, we recognize only \textbf{dc:a9:04:91:42:b9}: 
it's a \textit{MacBook Pro} in the same home. During the 5 minutes of the capture, this device was 
being used for smart-working purposes.s

\begin{figure}[h!]
    \centering
    \includegraphics[width=\linewidth]{/Users/lucaferraro/Desktop/PoliMi/First_year/Wireless_networks/Wirelesss_internet/Wireless_internet_project/Graphs/Multimedia_internet_bytes_packets.png}
    \caption{Multimedia internet lecture: data and packets exchanged.}
    \label{fig:Multimedia internet lecture: data and packets exchanged.}
\end{figure}




\subsection{Launch time capture.}
%Luca