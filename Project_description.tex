\subsection{Program setup}
To launch and use the program, first some pre-requisites have to be meet:
\begin{itemize}
    \item install pyshark library in order to analyze the capture file 
    \item install numpy library to perform some numerical analysis
    \item install matplotlib library to visualize plots
\end{itemize}
The program can run in 3 different modes selected by adding an attribute to the standard 
python call, using the command \texttt{python Traffic\_analyzer.py -file "PATH TO A FILE"} 
the program will scan the file provided after the \textit{-file} parameter. In this mode the program
will open the selected capture (in a .pcap/.pcapng format) and start scanning all the packets in 
order to obtain some information. 

The second mode is the live capturing mode in which, using the command \texttt{sudo python
Traffic\_analyzer.py -live "INTERFACE" "DURATION"} the program will first start a 
capture on the given interface,that has to be enabled to work in monitor mode, for a given 
amount of time provided with the "DURATION" attribute, will save this capture and will work
on it. With this mode administrator privileges are required to allow tshark to start the 
capture in monitor mode.  

The last mode is a default mode, launched using the command \texttt{python Traffic\_analyzer.py},
which will perform the analysis on a default capture inserted in the code. This mode is mostly 
used as a debug tool but can also be useful to understand how the output will look like.

\subsection{Program structure}
To obtain information from a capture, the code will run a for loop on all the packets in the 
capture file and will analyze only those which can be useful for our purpose. To select the 
useful the following filter is used:

\begin{lstlisting}[language=Python, caption=Packet filter]
    (int(packet.wlan.fc_type) == 2) and 
    ((int(packet.wlan.fc_subtype) >= 0 and 
        int(packet.wlan.fc_subtype) <= 3)) or
    (int(packet.wlan.fc_subtype) >= 8 and 
    int(packet.wlan.fc_subtype) <= 11)
\end{lstlisting}

it will select all the \textbf{data} frames (wlan.fc\_type == 2) and among those it will 
select the ones actually containing data or QoS data excluding null packets or ACKs. From 
those packets it will extract the destination address and will add it to a dictionary. 