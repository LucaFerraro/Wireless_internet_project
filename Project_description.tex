\subsection{Program setup}
To launch and use the program, first some pre-requisites have to be met:
\begin{itemize}
    \item install \textit{pyshark} library in order to analyze the capture file.
    \item install \textit{numpy} library to perform some numerical analysis.
    \item install \textit{matplotlib} library to visualize plots.
\end{itemize}
The program can run in 3 different modes selected by adding an attribute to the standard 
python call.\\
Using the command \texttt{python Traffic\_analyzer.py -file "PATH TO A FILE"} 
the program will scan the file provided after the \textit{-file} parameter. In this mode the program
will open the selected capture (in a .pcap/.pcapng format) and start scanning all the packets in 
order to obtain some information. 

The second mode is the live capturing mode: using the command \texttt{sudo python
Traffic\_analyzer.py -live "INTERFACE" "DURATION"}, the program will first start a 
capture on the given interface, that has to be enabled to work in monitor mode, for a given 
amount of time provided with the "DURATION" attribute, will save this capture and will work
on it. With this mode administrator privileges are required to allow \textit{tshark} to start the 
capture in monitor mode.  

The last mode is a default mode, launched using the command \texttt{python Traffic\_analyzer.py},
which will perform the analysis on a default capture inserted in the code. This mode is mostly 
used as a debug tool but can also be useful to understand how the output of the program looks like.

\subsection{Program structure}
To obtain information from a capture, the code will run a \textit{for} loop on all the packets in the 
capture file and will analyze only those which can be useful for our purpose. To select the 
useful the following filter is used:

\begin{lstlisting}[language=Python, caption=Packet filter]
    (int(packet.wlan.fc_type) == 2) and 
        ((int(packet.wlan.fc_subtype) >= 0 and 
            int(packet.wlan.fc_subtype) <= 3)) or
        (int(packet.wlan.fc_subtype) >= 8 and 
        int(packet.wlan.fc_subtype) <= 11)
\end{lstlisting}
Thi filter selects all the \textit{data} frames (\texttt{wlan.fc\_type == 2}) and among those it will 
select the ones actually containing \textit{data} or \textit{QoS data}, excluding null packets or ACKs. 
From those packets, the program will extract the destination address and add it to a dictionary
which will store the number of bytes in the packet, extracted from its \texttt{packet.data.len} field, and 
the number of packets received and transmitted by this MAC during the capture.
Then it will try to extract, if present, the source address and store information the same 
way. During the development phase we noticed that the source address (\texttt{wlan.packet.sa}) was 
not always present so we decided to check if it was present in order to avoid errors.

Before the end of the loop, the lists for the cumulative traffic curves are updated. To 
effectively print those curves we had to divide the time in discrete intervals of length T and 
check if the \texttt{packet.frame\_info.time\_relative} was inside the current interval. If the
interval is correct, the counters are updated, otherwise a new time slot is created.

\begin{lstlisting}[language=Python, caption=Traffic curves setup]
    # Updating traffic curve:
    if (t_capture >= n*T): # if t_capture in [(n+1)T, (n+2)T]
        traffic_in.append(nBytes_rx)
        traffic_out.append(nBytes_tx)
        n = n + 1

    else: # if t_capture in [nT, (n+1)T]
        traffic_in[n-1] = traffic_in[n-1] + nBytes_rx
        traffic_out[n-1] = traffic_out[n-1] + nBytes_tx

    nData = nData + nBytes_rx
    nPacket = nPacket + 1
\end{lstlisting}

After the for loop the program will have all the lists ready to print the retrieved data.\\ 
First of all we print the total capture duration, the total number of bytes processed and the total 
number of packets exchanged. Than for each MAC address registered during the processing of the capture, 
we print the vendor to which it belongs to, retrieved using the IEEE standard file, and all the 
data retrieved during the capture. Those data are also plotted in three graphs using matplotlib.
The first two graphs show some histograms about received and transmitted bytes and packets for
some of the registered MAC addresses; we decided not to insert all of them because we wanted to plot 
only relevant information. To do this, we used a variable called \texttt{plot\_ratio}: only the data relavite to 
the \textit{MAC} addresses that have transmitted or received at least plot\_ratio\% packets compared to the 
\textit{MAC} address which sent or received the maximum amount in the capture are plotted in the histograms.\\
The last graphs show the traffic curve of the the received and the transmitted packets.
